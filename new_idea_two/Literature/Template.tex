% Template for IGARSS-2020 paper; to be used with:
%          spconf.sty  - LaTeX style file, and
%          IEEEbib.bst - IEEE bibliography style file.
% --------------------------------------------------------------------------
\documentclass{article}
\usepackage{spconf,amsmath,epsfig}
\usepackage{subcaption}
\usepackage{multirow}
% Example definitions.
% --------------------
\def\x{{\mathbf x}}
\def\L{{\cal L}}

% Title.
% ------
\title{USING THE GAF TRANSFORM AND MODIS TIME-SERIES TO PERFORM LANDCOVER CLASSIFICATION AND CHANGE DETECTION}
%
% Single address.
% ---------------
\name{T.L. Grobler$^{\dagger}$, W. Kleynhans$^{\star}$ and B.P. Salmon$^{\ddagger}$}
\address{$\dagger$Dept of Mathematical Sciences, Computer Science Division, Stellenbosch University,\\ Private Bag X1, 7602 Matieland, South Africa\\
$\star$Department of Electrical, Electronic and Computer Engineering University of Pretoria,\\
Pretoria 0002, South Africa\\
${\ddagger}$School of Engineering, University of Tasmania,
Hobart, TAS 7001, Australia}
%
% For example:
% ------------
%\address{School\\
%	Department\\
%	Address}
%
% Two addresses (uncomment and modify for two-address case).
% ----------------------------------------------------------
%\twoauthors
%  {A. Author-one, B. Author-two\sthanks{Thanks to XYZ agency for funding.}}
%	{School A-B\\
%	Department A-B\\
%	Address A-B}
%  {C. Author-three, D. Author-four\sthanks{The fourth author performed the work
%	while at ...}}
%	{School C-D\\
%	Department C-D\\
%	Address C-D}
%
\begin{document}
%\ninept
%
\maketitle
%
\begin{abstract}
The abstract should appear at the top of the left-hand column of text, about
0.5 inch (12 mm) below the title area and no more than 3.125 inches (80 mm) in
length.  Leave a 0.5 inch (12 mm) space between the end of the abstract and the
beginning of the main text.  The abstract should contain about 100 to 150
words, and should be identical to the abstract text submitted electronically
along with the paper cover sheet.  All manuscripts must be in English, printed
in black ink.
\end{abstract}
%
\begin{keywords}
Moderate Resolution Imaging Spetroradiometer, Gramian Angular Fields, landcover classification and change detection, settlement expansion
\end{keywords}
%
\section{Introduction}
\label{sec:intro}
The field of computer vision has changed dramatically over the last few years due to the success of Convolutional Neural Networks (CNNs) and deep learning. Recently, it was proposed to encode time-series as images so that the capability of CNNs to perform image classification could also be applied to the problem of time-series classification. One possible time-series encoding scheme is the Gramian Angular Field (GAF) transform. GAF images are formed using polar coordinates and ultimately take the form of a Gramian matrix, whose entries are the trigonometric sum of observations recorded at different time indices. This encoding scheme has also been succesfully applied within the remote sensing field. Dias et al used the GAF transform (and other transforms), Moderate Resolution Imaging Spectroradiometer (MODIS) time-series and existing CNN models to succesfully detect eucalyptus plantations. Their CNN based solution outperformed the conventional approaches they compared against. In this paper we will investigate whether MODIS time-series and the GAF transfrom can be utilized for detecting settlement expansion. Settlement expansion is one of the most pervasive forms of landcover change in sub saharan Africa. It is, therefore, of the utmost importance to find strategies and algorithms that will allow us to effectively monitor the growth of settlements in Africa. In this paper, however, we will restrict ourselves to a case study in the Gauteng province of South Africa. In addition to investigating another use case than Dias et al we also extend their work in the following ways:
\begin{itemize}
 \item We do not restrict ourselves to only investigating the problem of land cover classification, we also investigate whether the GAF transfrom can be utilized for detecting landcover changes.
 \item We show that one does not necessarily require deep learning techniques when one employes the GAF transform to perform land cover classification and change detection. A conventional classifier in conjunction with the GAF transform is already capable of discerning highly separable landcover classes from one another (it can also succesfully detect pervasive land cover changes).
\end{itemize}
  

\section{Data Description}
\label{sec:data}
In this paper we make use of a dataset composed of MODIS NDVI (Natural Difference Vegetaion Index) time-series/pixels. The NDVI time-series was constructed from band one and two of the MODIS MCD43A4 product. The study area we considered was approximately 230~km$^2$ in Gauteng, South Africa. The time-period considered was between January 2001 and March 2009. The temporal cadence of the data is every eight days (i.e. 45 observations a year). Each time-series in the dataset consists of 368 observations and its associated spatial resolution is 500~m by 500~m. The dataset contains MODIS pixels from two landcover types: vegetation ($v$) and settlement ($s$). A MODIS pixel was assigned the label $v$ if it contained more than 90\% vegetation, whereas a pixel was assigned the label $s$ if it contained more than 50\% buildings. The dataset contains 1105 MODIS pixels: 592 vegetation pixles, 333 settlement pixels, and 180 pixles that changed from vegetation to settlement. Pixels were selected by visually inspecting two high-resolution Syst\`{e}me Probatoried d'Observation de la Terra (SPOT) images from 2001 and 2009. Only pixels that according to the SPOT images did not change or changed from vegtation to settlement were selected.

\section{Gramian Angular Field}
\label{sec:GAF}
The GAF transform makes it possible to encode time-series as images. The resulting images can then be classified using image based deep learning methods. Let $\mathbf{x} = \{x_1,x_2,\cdots,x_N\}$ be a time series consisting of $N$ points. Rescale $\mathbf{x}$ between -1 and 1. The GAF of $\mathbf{x}$ is then defined as: 
\begin{equation}
\mathbf{G} = \begin{pmatrix}
\cos(\phi_1+\phi_1) & \cdots & \cos(\phi_1+\phi_N)\\
\vdots& \ddots &\vdots\\
\cos(\phi_N+\phi1)&\cdots&\cos(\phi_N+\phi_N)
\end{pmatrix},\nonumber
\end{equation}
where $\phi_j = \arccos(x_j)$.
  
\section{Methodology}
\label{sec:exp}

We conducted two main experiments. We investigated whether the GAF transfrom could be effectively utilized for encoding NDVI time-series as images for the purpose of land cover classification and change detection. These two experiments are respectively discussed in Section~\ref{sec:gaf_class} and Section~\ref{sec:gaf_change}. The two benchmarking approaches we employed to compare against are described in Section~\ref{sec:pca} and Section~\ref{sec:diff}. The results obtained from the experiments below are presented in Table~\ref{tab:cm} and Fig.~\ref{fig:results} and is discussed further in Section~\ref{sec:results}        
 %We benchmarked our classification results against a well documented approach within literature.

\subsection{PCA: Land cover classification}
\label{sec:pca}
The procedure as outlined below was repeated 10 times to ensure the reliability of the results.  We first extracted the largest two principal components from the non-chaning NDVI time-series within our dataset (see Section~\ref{sec:data}). The reduced dataset so obtained was then randomly split into a test and a training set. About 50\% of the data was used for training and 50\% for testing. We then used our training set to train \textsc{SCIKIT-LEARN}'s logistic regression classifier. We did not deem it neccecarry to perform any kind of hyperparameter tuning as the deault settings produced adequate results. The accuracy of our classifier was then tested by using the test set. 

\subsection{GAF: Land cover classification}
\label{sec:gaf_class}
The procedure as outlined below was repeated 10 times to ensure the reliability of the resutls. The time-series of the pixels in our dataset that did not change were encoded using the GAF transform (see Section~\ref{sec:GAF}). This non-changing encoded dataset was then randomly split into a training set and a test set. About 50\% of the data was used for training and 50\% for testing. The test set was then used to train \textsc{SCIKIT-LEARN}'s logistic regression classifier. We used the classsifier's default settings. We did not deem it neccecarry to perform any kind of hyperparameter tuning as the deault settings produced adequate results. We then tested the accuracy of our classifier by using the test set. 

\subsection{Band differencing: Land cover change detection}
\label{sec:diff}
The best case performance of the band differencing algorithm was determined. We used the non-changing vegetation pixels and the real-change pixels from Section~\ref{sec:data} to conduct our experiment. The approach works by first reducing the noise contained within the data by filtering the NDVI time-series. The annual sum of each fikltered time-series is then determined. The difference between these sums are then computed. The $z$-score of each difference is then computed along the spatial axis. The maximum $z$-score associated with each time-series is then selected. If this $z$-score is larger than a predefined threshold then it is assumed that the pixel in question had changed over the observation period. Grid search was employed to obtain the best performing threshold.


\subsection{GAF: Land cover change detection}
\label{sec:gaf_change}
The procedure as outlined below was repeated 10 times to ensure the reliability of the resutls. We start by randomly splitted our vegetation pixels into a partial training set (200 pixels), a partial testing set (200 pixles) and a temporary set (192 pixels). All of our settlement pixels were added to a temporary settlement set (333 pixels). All the real change pixels were added to a temporary real change set (180 pixels). We then created a synthetic change set (200 pixels) by linearly splicing together random pixels from our temporary vegetation set (192 pixels) and our temporary settlement set (333) over a six month period. The change points were uniformly chosen from the set [1,300]. The final training  set (400 pixels) is then constructed by lumping together the vegetation training set (200 pixels) and the synthetic change set (200 pixels). The final test set (380 pixels) is then created by lumping together the vegetation testing set (200 pixles) and the real change set (180 pixels). The time-series of the pixels are then encoded using the GAF transform. \textsc{SCIKIT-LEARN}'s logistic regression classifier. We used used the classsifier's default settings. We did not deem it neccecarry to perform any kind of hyperparameter tuning as the deault settings produced adequate results. We then tested the accuracy of our classifier by using the test set. The results we obtained is presented in Table and Figure and is discussed further in Section

\section{Results}
\label{sec:results}
\begin{table}
\begin{tabular}{ |l|c c c| }
\hline
%\multicolumn{3}{ |c| }{Team sheet} \\
& & v & s\\
\hline
\multirow{2}{*}{PCA} & \emph{v} & 93.63$\pm$1.58&6.37\\
& \emph{s}& 0.6& 99.41\%$\pm$0.39\\
\hline
\multirow{2}{*}{GAF} & \emph{v} & 99.42$\pm$0.58&0.58\\
& \emph{s}& 0.77& 99.23\%$\pm$0.64\\
\hline
\hline
& & nc & c\\
\hline
\multirow{2}{*}{Differencing} & \emph{nc} & 85.47 & 14.53\\
& \emph{c}& 16.11 & 83.89\\
\multirow{2}{*}{GAF} & \emph{nc} & 99.2$\pm$0.71 & 0.8\\
& \emph{c}& 0.01 & 99.9$\pm$0.17\\
%Goalkeeper & GK & Paul Robinson \\ \hline
%\multirow{4}{*}{Defenders} & LB & Lucas Radebe \\
% & DC & Michael Duburry \\
% & DC & Dominic Matteo \\
% & RB & Didier Domi \\ \hline
%\multirow{3}{*}{Midfielders} & MC & David Batty \\
% & MC & Eirik Bakke \\
% & MC & Jody Morris \\ \hline
%Forward & FW & Jamie McMaster \\ \hline
%\multirow{2}{*}{Strikers} & ST & Alan Smith \\
% & ST & Mark Viduka \\
\hline
\end{tabular}
\caption{xxx}
\label{tab:cm}
\end{table}

\begin{figure}
 \includegraphics[width=0.48\textwidth]{results-crop.pdf}
 \label{fig:results}
 \caption{Test}
\end{figure}

%Given this background, let us list the  contributions that we make in this paper: 
%\begin{itemize}
% \item Our indepenent study corroborates the study made by Dias et al, i.e. we have confirmed that the GAF transform is a usefull encoding scheme that can be used for pixel-based land cover classification. 
% \item We show that the GAF transform can be utilized for detecting settlement expasion.
% \item We show that the GAF transform can not only be used for pixel based land cover classification, but also for detecting land cover change.
% \item We show that is the classification problem at hand is seperable enough
%\end{itemize}



In addition to investigating wheter the GAF transform can be utilized for detecting settlement expansion, the current paper also makes the following unique contributions:

we also Let us now expand on the unique contributions of the current paper in a bit Expanding on the , the unique contributions of the current paper are:
\begin{itemize}
 \item 
\end{itemize}




GAF images are rep-resented as a Gramian matrix where each element is thetrigonometric sum (i.e., superposition of directions) be-tween different time intervals


such an approach is to use the GAF transform to encode time-series as images for classification The  transform  



As an example of it, CNNs can easily distinguish photographs of dogs and cats  


These guidelines include complete descriptions of the fonts, spacing, and
related information for producing your proceedings manuscripts. Please follow
them and if you have any questions, direct them to Conference Management
Services, Inc.: Phone +1-979-846-6800 or Fax +1-979-846-6900 or email
to \verb+papers@igarss2020.org+.

\section{Formatting your paper}
\label{sec:format}

All printed material, including text, illustrations, and charts, must be kept
within a print area of 7 inches (178 mm) wide by 9 inches (229 mm) high. Do
not write or print anything outside the print area. The top margin must be 1
inch (25 mm), except for the title page, and the left margin must be 0.75 inch
(19 mm).  All {\it text} must be in a two-column format. Columns are to be 3.39
inches (86 mm) wide, with a 0.24 inch (6 mm) space between them. Text must be
fully justified.

\section{PAGE TITLE SECTION}
\label{sec:pagestyle}

The paper title (on the first page) should begin 1.38 inches (35 mm) from the
top edge of the page, centered, completely capitalized, and in Times 14-point,
boldface type.  The authors' name(s) and affiliation(s) appear below the title
in capital and lower case letters.  Papers with multiple authors and
affiliations may require two or more lines for this information.

\section{TYPE-STYLE AND FONTS}
\label{sec:typestyle}

To achieve the best rendering in the proceedings, we
strongly encourage you to use Times-Roman font.  In addition, this will give
the proceedings a more uniform look.  Use a font that is no smaller than ten
point type throughout the paper, including figure captions.

In ten point type font, capital letters are 2 mm high.  If you use the
smallest point size, there should be no more than 3.2 lines/cm (8 lines/inch)
vertically.  This is a minimum spacing; 2.75 lines/cm (7 lines/inch) will make
the paper much more readable.  Larger type sizes require correspondingly larger
vertical spacing.  Please do not double-space your paper.  True-Type 1 fonts
are preferred.

The first paragraph in each section should not be indented, but all the
following paragraphs within the section should be indented as these paragraphs
demonstrate.

\section{MAJOR HEADINGS}
\label{sec:majhead}

Major headings, for example, "1. Introduction", should appear in all capital
letters, bold face if possible, centered in the column, with one blank line
before, and one blank line after. Use a period (".") after the heading number,
not a colon.

\subsection{Subheadings}
\label{ssec:subhead}

Subheadings should appear in lower case (initial word capitalized) in
boldface.  They should start at the left margin on a separate line.
 
\subsubsection{Sub-subheadings}
\label{sssec:subsubhead}

Sub-subheadings, as in this paragraph, are discouraged. However, if you
must use them, they should appear in lower case (initial word
capitalized) and start at the left margin on a separate line, with paragraph
text beginning on the following line.  They should be in italics.

\section{PRINTING YOUR PAPER}
\label{sec:print}

Print your properly formatted text on high-quality, 8.5 x 11-inch white printer
paper. A4 paper is also acceptable, but please leave the extra 0.5 inch (12 mm)
empty at the BOTTOM of the page and follow the top and left margins as
specified.  If the last page of your paper is only partially filled, arrange
the columns so that they are evenly balanced if possible, rather than having
one long column.

In LaTeX, to start a new column (but not a new page) and help balance the
last-page column lengths, you can use the command ``$\backslash$pagebreak'' as
demonstrated on this page (see the LaTeX source below).

\section{PAGE NUMBERING}
\label{sec:page}

Please do {\bf not} paginate your paper.  Page numbers, session numbers, and
conference identification will be inserted when the paper is included in the
proceedings.

\section{ILLUSTRATIONS, GRAPHS, AND PHOTOGRAPHS}
\label{sec:illust}

Illustrations must appear within the designated margins.  They may span the two
columns.  If possible, position illustrations at the top of columns, rather
than in the middle or at the bottom.  Caption and number every illustration.
All illustrations should be clear wwhen printed on a black-only printer. Color
may be used.

Since there are many ways, often incompatible, of including images (e.g., with
experimental results) in a LaTeX document, below is an example of how to do
this \cite{Lamp86}.

% Below is an example of how to insert images. Delete the ``\vspace'' line,
% uncomment the preceding line ``\centerline...'' and replace ``imageX.ps''
% with a suitable PostScript file name.
% -------------------------------------------------------------------------


\begin{figure*}[h] 
  \begin{subfigure}[b]{0.5\linewidth}
    \centering
    \includegraphics[width=0.89\linewidth]{veg-crop.pdf} 
    \caption{Vegetation} 
    \label{fig7:a} 
    \vspace{6ex}
  \end{subfigure}%% 
  \begin{subfigure}[b]{0.5\linewidth}
    \centering
    \includegraphics[width=0.89\textwidth]{bwt-crop.pdf} 
    \caption{Settlement} 
    \label{fig7:b} 
    \vspace{6ex}
    \end{subfigure} 
    %\vspace{13ex}
  \begin{subfigure}[b]{0.5\linewidth}
    \centering
    \includegraphics[width=0.89\textwidth]{sim_c-crop.pdf} 
    \caption{Simulated Change} 
    \label{fig7:c} 
  \end{subfigure}%%
  \begin{subfigure}[b]{0.5\linewidth}
    \centering
    \includegraphics[width=0.89\textwidth]{real_c-crop.pdf} 
    \caption{Real Change} 
    \label{fig7:d} 
  \end{subfigure} 
  \label{fig7} 
  \caption{The classification accuracy (\%) results obtained if we employ the supervised (top left) and the unsupervised (top right) sequential classification approaches presented in Section~\ref{sec:sprt} to classify (cluster) the dataset presented in Section~\ref{sec:data}. The classification accuracy results obtained by applying $k$-means (bottom left) and GMM (bottom-right) to the the dataset presented in Section~\ref{sec:data} are also presented. The supervised approach outperforms the unsupervised approach as is expected. Note, the unsupervised sequential approach, however, outperforms traditional $k$-means.}
  \label{fig:results}
\end{figure*}






% To start a new column (but not a new page) and help balance the last-page
% column length use \vfill\pagebreak.
% -------------------------------------------------------------------------
\vfill
\pagebreak




\section{FOOTNOTES}
\label{sec:foot}

Use footnotes sparingly (or not at all!) and place them at the bottom of the
column on the page on which they are referenced. Use Times 9-point type,
single-spaced. To help your readers, avoid using footnotes altogether and
include necessary peripheral observations in the text (within parentheses, if
you prefer, as in this sentence).


\section{COPYRIGHT FORMS}
\label{sec:copyright}

You must also electronically sign the IEEE copyright transfer
form when you submit your paper. We {\bf must} have this form
before your paper can be sent to the reviewers or published in
the proceedings. The copyright form is provided through the IEEE
website for electronic signature. A link is provided upon
submission of the manuscript to enter the IEEE Electronic
Copyright Form system.

\section{REFERENCES}
\label{sec:ref}

List and number all bibliographical references at the end of the paper.  The references can be numbered in alphabetic order or in order of appearance in the document.  When referring to them in the text, type the corresponding reference number in square brackets as shown at the end of this sentence \cite{C2}.

% References should be produced using the bibtex program from suitable
% BiBTeX files (here: strings, refs, manuals). The IEEEbib.bst bibliography
% style file from IEEE produces unsorted bibliography list.
% -------------------------------------------------------------------------
\bibliographystyle{IEEEbib}
\bibliography{strings,refs}

\end{document}
